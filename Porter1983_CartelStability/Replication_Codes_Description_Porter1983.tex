\documentclass[12pt]{article}
\usepackage[margin=1in]{geometry} 
\usepackage{amsmath,amsthm,amssymb,amsfonts}
\usepackage{enumerate,graphicx,float}
\usepackage{listings,lscape}

\newcommand{\N}{\mathbb{N}}
\newcommand{\Z}{\mathbb{Z}}
\usepackage{bm}
\usepackage{booktabs}
 
\newenvironment{question}[2][Question]{\begin{trivlist}
\item[\hskip \labelsep {\bfseries #1}\hskip \labelsep {\bfseries #2.}]}{\end{trivlist}}
 
\begin{document}
\title{\textbf{Replication Codes Description: Porter (1983)}}
\author{ZHANG Donglei \\ \textit{CUHK Business School}}
\maketitle 
%---------------------------------------------------------------
%---------------------------------------------------------------
\section{Abstract}
\begin{itemize}
\item This documents may help you understand the replication codes for the paper by R.H. Porter in 1983 about the cartel stability.
\item Most of the codes are written by Matlab, and the others are written by Stata.
\item You just need to run \textit{mlepn.m} to obtain the final result.
\end{itemize}

%-------------------------------------------------------------------
\section{Files}
\subsection{Stata}
\begin{itemize}
\item \textbf{datageneration.do} This do-file generates necessary variables used in the later process. This file also generates necessary values to replicate Table 2 Summary Statistics. (import: ``\textit{Data/jec.txt}"; output: ``\textit{Data/jecnew.dta}")

\item \textbf{probit.do} This do-file runs Probit regression to provide the initial value of $\{w_1,...,w_T\}$ for iterations. But when using this file to obtain the initial value, the final result of MLE will be quite different from the value on Table 3 as well as the result by the code provider. (import: ``\textit{Data/jecnew.dta}"; output: ``\textit{Data/wght0.csv}")

\item \textbf{tsls.do} This do-file runs two-stage least squares regression on demand side and supply side. The result of this regression consists the first column of Table 3 on page 309. But I don't why this estimation result is quite different from the value on Table 3. So I do not use this result as the initial value in MLE.
\end{itemize} 

\subsection{Matlab}
\begin{itemize}
\item \textbf{tsls.m} (script): The file is much similar with \textit{tsls.do} to run two-stage least squares regression on demand side and supply side. The result of this regression consists the first column of Table 3 on page 309. And it also provides the initial value in MLE to search the best parameters. (import: ``\textit{Data/jec.txt}"; output: ``\textit{Data/param.mat}")

\item \textbf{mlepn.m} (script): This file contains the codes of maximum log likelihood estimation. (import: ``\textit{Data/jec.txt}", ``\textit{Data/param.mat}"; output: ``\textit{Data/jecnew.mat}", results on the second column of Table 3)
	\begin{itemize}
	\item \textbf{L1.m} (function): This function provides the maximum log likelihood function (set negative to find its minimum value). (import: ``\textit{Data/jecnew.mat}", ``\textit{Data/wght0.csv}"; output: the formula of maximum log likelihood function)
		\begin{itemize}
		\item \textbf{lmda.m} (function): This function computes the weights for the expected log maximum likelihood in the later iterations (update $\lambda_t$ and $w_t$ series). (import: ``\textit{Data/jecnew.mat}"; output: the value of $\lambda_t$)
		\end{itemize}
    \item \textbf{L0.m} (function): This file is similar with \textit{mlepn.m} which contains the codes of log maximum likelihood estimation, but we here let the coefficient of $I_t$ (that is $po$) equal to 0. (import: ``\textit{Data/param.mat}", ``\textit{Data/jecnew0.mat}"; output: the optimal value of the log likelihood function $L_0$).
		\begin{itemize}
		\item \textbf{LL0.m} (function): This file is similar with \textit{L1.m} whichprovides the maximum likelihood function (set negative to find its minimum value), but we here let the coefficient of $I_t$ (that is $po$) equal to 0 and do not need to calculate $\lambda_t$ and $w_t$ series. (import: ``\textit{Data/jecnew.mat}"; output: the formula of log maximum likelihood function)
		\end{itemize}
	\end{itemize}
\end{itemize}


%-------------------------------------------------------------------
\section{Math Appendix: Maximum Likelihood Function}

\textbf{Reference:} [DM] Russell Davidson and James G. MacKinnon (1999). Econometric Theory and Methods.

\subsection{Full-Information Maximum Likelihood Function }
\[\log L (I_1,...,I_T) = -\frac{gn}{2}\log 2\pi -\frac{n}{2}\log \left | \bm{\Sigma} \right |  +n\log \left | det \bm{\Gamma} \right | - \frac{1}{2}(\bm{y_\cdot}-\bm{X_\cdot \beta_\cdot})^T(\bm{\Sigma}^{-1}\otimes \bm{I_n})(\bm{y_\cdot}-\bm{X_\cdot \beta_\cdot})\]
(Davidson \& MacKinnon, 12.80)
\lstset{language=Matlab} 
\begin{lstlisting}
-------------------------------------------------------------------
GM=[1 -beta(nb1d+1+nb1s+1);-beta(nb1d+1) 1];
SG=[(beta(nb-2)) beta(nb);beta(nb) (beta(nb-1))]; 
B=[beta(1:nb1d)' 0 0 0 0 0; beta(nb1d+1+1) 0 beta(nb1d+1+2:nb1d+1+nb1s)']';
betadot=beta(1:nb1d+1+nb1s+1);
ydot=[Y(:,1);Y(:,2)];
W(:,size(W,2))=pn; %po replaced with the new weights
Z2(:,size(Z2,2))=pn; %po replaced with the new weights
Xdot=blkdiag([Z1 lngr],[Z2 lnQ]);
%The negative log likelihood
kost=-(G*T/2)*log(2*pi)-(T/2)*log(det(SG))+T*(log(abs(det(GM)))); 
f=-kost+.5*((ydot-Xdot*betadot)'*kron(inv(SG),eye(T))*(ydot-Xdot*betadot)); 
-------------------------------------------------------------------
\end{lstlisting}
\ \ \

\subsection{The covariance matrix of $\bm{\hat{\beta}_\cdot^{ML}}$ }
\[\hat{Var}(\bm{\hat{\beta}_\cdot^{ML}})=(\bm{X_\cdot}^T(\bm{\hat{B}_{ML}},\bm{\hat{\Gamma}_{ML}}) ( \bm{\hat{\Sigma}_{ML}^{-1} \otimes \bm{I_n}  })\bm{X_\cdot} (\bm{\hat{B}_{ML}},\bm{\hat{\Gamma}_{ML}}))^{-1} \text{      (DM, 12.89) }\]
where
\[\bm{X_\cdot} (\bm{\hat{B}_{ML}},\bm{\hat{\Gamma}_{ML}}) = 
\begin{bmatrix}
\bm{Z_\cdot} & \hat{\bm{Y_\cdot}}(\bm{\hat{B}_{ML}},\bm{\hat{\Gamma}_{ML}})
\end{bmatrix} \text{     (DM, 12.55)}\]
where
\[\hat{\bm{Y_\cdot}}(\bm{\hat{B}},\bm{\hat{\Gamma}}) \bm{\hat{\Gamma}} = \bm{W\hat{B}} \text{      (DM, 12.68)}\]
\[\hat{\bm{Y_\cdot}}(\bm{\hat{B}},\bm{\hat{\Gamma}}) = \bm{W\hat{B}} \bm{\hat{\Gamma}} ^{-1} \text{      (DM, 12.70)}\]
Thus
\[\bm{X_\cdot}^{new} = 
\begin{bmatrix}
\bm{Z_\cdot} & \bm{W\hat{B}} \bm{\hat{\Gamma}} ^{-1}
\end{bmatrix}\]

\lstset{language=Matlab} 
\begin{lstlisting}
-------------------------------------------------------------------
Ydot_new=W*B*(inv(GM)); 
	%Y*GM=W*B+U -> Y=W*B*(GM^-1)+V
Xdot_new=blkdiag([Z1 Ydot_new(:,2)],[Z2 Ydot_new(:,1)]); 
	%Xt=[Zt Yt], since y=X*beta+u=Z*beta1+Y*beta2+u
se=sqrt(diag(inv(Xdot_new'*(kron(inv(SG),eye(T)))*Xdot_new))); 
	%Calculate the standard errors of estimaterd coefficients
se=[se;sqrt(diag(2.*kron(SG,SG)./T))]; 
	%Add estimated standard errors to the vector se
se(length(se)-1)=[];  
	% [shat1 shat12; shat12 shat2] -> [shat1 shat12 shat12 shat2]
	% Delete one of the shat12 to make the vector dimension=37
-------------------------------------------------------------------
\end{lstlisting}

%---------------------------------------------------------------
\newpage
\section{Index of \textit{beta}}
\begin{table}[h]
\centering
\caption{Index of \textit{beta1} \& \textit{beta2}}
\vspace{3mm}
\begin{tabular}{c|c|c|c}
\hline
\multicolumn{2}{c|}{\textbf{\textit{beta1}}} & \multicolumn{2}{|c}{\textbf{\textit{beta2}}} \\ \hline
\textbf{Index}   & \textbf{Coefficient of}  & \textbf{Index}   & \textbf{Coefficient of}    \\ \hline
1       & C               & 1       & C                 \\
2       & L               & 2-5     & DM1-4             \\
3-14    & month1-12       & 6       & po                \\
15      & lngr            & 7-18    & month1-12         \\
        &                 & 19      & lnQ               \\ \hline
\end{tabular}
\end{table}

\begin{table}[h]
\centering
\caption{Index of \textit{beta3}} 
\vspace{3mm}
\begin{tabular}{c|c|c}
\hline
\multicolumn{3}{c}{\textbf{\textit{beta3}}}         \\ \hline
\textbf{Index} & \textbf{Coefficient of} & \textbf{Equation} \\ \hline
1     & C              &          \\
2     & L              & Demand   \\
3-14  & month1-12      &          \\
15    & lngr           &          \\ \hline
16    & C              &          \\
17-28 & month1-12      &          \\
29-32 & DM1-4          & Supply   \\
33    & po             &          \\
34    & lnQ            &          \\ \hline
35    & shat1          &          \\
36    & shat2          & Standard error     \\
37    & shat12         &          \\
\hline
\end{tabular}
\end{table}
%---------------------------------------------------------------
%---------------------------------------------------------------
\end{document}